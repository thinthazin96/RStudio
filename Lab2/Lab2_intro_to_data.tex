% Options for packages loaded elsewhere
\PassOptionsToPackage{unicode}{hyperref}
\PassOptionsToPackage{hyphens}{url}
\documentclass[
]{article}
\usepackage{xcolor}
\usepackage[margin=1in]{geometry}
\usepackage{amsmath,amssymb}
\setcounter{secnumdepth}{-\maxdimen} % remove section numbering
\usepackage{iftex}
\ifPDFTeX
  \usepackage[T1]{fontenc}
  \usepackage[utf8]{inputenc}
  \usepackage{textcomp} % provide euro and other symbols
\else % if luatex or xetex
  \usepackage{unicode-math} % this also loads fontspec
  \defaultfontfeatures{Scale=MatchLowercase}
  \defaultfontfeatures[\rmfamily]{Ligatures=TeX,Scale=1}
\fi
\usepackage{lmodern}
\ifPDFTeX\else
  % xetex/luatex font selection
\fi
% Use upquote if available, for straight quotes in verbatim environments
\IfFileExists{upquote.sty}{\usepackage{upquote}}{}
\IfFileExists{microtype.sty}{% use microtype if available
  \usepackage[]{microtype}
  \UseMicrotypeSet[protrusion]{basicmath} % disable protrusion for tt fonts
}{}
\makeatletter
\@ifundefined{KOMAClassName}{% if non-KOMA class
  \IfFileExists{parskip.sty}{%
    \usepackage{parskip}
  }{% else
    \setlength{\parindent}{0pt}
    \setlength{\parskip}{6pt plus 2pt minus 1pt}}
}{% if KOMA class
  \KOMAoptions{parskip=half}}
\makeatother
\usepackage{color}
\usepackage{fancyvrb}
\newcommand{\VerbBar}{|}
\newcommand{\VERB}{\Verb[commandchars=\\\{\}]}
\DefineVerbatimEnvironment{Highlighting}{Verbatim}{commandchars=\\\{\}}
% Add ',fontsize=\small' for more characters per line
\usepackage{framed}
\definecolor{shadecolor}{RGB}{248,248,248}
\newenvironment{Shaded}{\begin{snugshade}}{\end{snugshade}}
\newcommand{\AlertTok}[1]{\textcolor[rgb]{0.94,0.16,0.16}{#1}}
\newcommand{\AnnotationTok}[1]{\textcolor[rgb]{0.56,0.35,0.01}{\textbf{\textit{#1}}}}
\newcommand{\AttributeTok}[1]{\textcolor[rgb]{0.13,0.29,0.53}{#1}}
\newcommand{\BaseNTok}[1]{\textcolor[rgb]{0.00,0.00,0.81}{#1}}
\newcommand{\BuiltInTok}[1]{#1}
\newcommand{\CharTok}[1]{\textcolor[rgb]{0.31,0.60,0.02}{#1}}
\newcommand{\CommentTok}[1]{\textcolor[rgb]{0.56,0.35,0.01}{\textit{#1}}}
\newcommand{\CommentVarTok}[1]{\textcolor[rgb]{0.56,0.35,0.01}{\textbf{\textit{#1}}}}
\newcommand{\ConstantTok}[1]{\textcolor[rgb]{0.56,0.35,0.01}{#1}}
\newcommand{\ControlFlowTok}[1]{\textcolor[rgb]{0.13,0.29,0.53}{\textbf{#1}}}
\newcommand{\DataTypeTok}[1]{\textcolor[rgb]{0.13,0.29,0.53}{#1}}
\newcommand{\DecValTok}[1]{\textcolor[rgb]{0.00,0.00,0.81}{#1}}
\newcommand{\DocumentationTok}[1]{\textcolor[rgb]{0.56,0.35,0.01}{\textbf{\textit{#1}}}}
\newcommand{\ErrorTok}[1]{\textcolor[rgb]{0.64,0.00,0.00}{\textbf{#1}}}
\newcommand{\ExtensionTok}[1]{#1}
\newcommand{\FloatTok}[1]{\textcolor[rgb]{0.00,0.00,0.81}{#1}}
\newcommand{\FunctionTok}[1]{\textcolor[rgb]{0.13,0.29,0.53}{\textbf{#1}}}
\newcommand{\ImportTok}[1]{#1}
\newcommand{\InformationTok}[1]{\textcolor[rgb]{0.56,0.35,0.01}{\textbf{\textit{#1}}}}
\newcommand{\KeywordTok}[1]{\textcolor[rgb]{0.13,0.29,0.53}{\textbf{#1}}}
\newcommand{\NormalTok}[1]{#1}
\newcommand{\OperatorTok}[1]{\textcolor[rgb]{0.81,0.36,0.00}{\textbf{#1}}}
\newcommand{\OtherTok}[1]{\textcolor[rgb]{0.56,0.35,0.01}{#1}}
\newcommand{\PreprocessorTok}[1]{\textcolor[rgb]{0.56,0.35,0.01}{\textit{#1}}}
\newcommand{\RegionMarkerTok}[1]{#1}
\newcommand{\SpecialCharTok}[1]{\textcolor[rgb]{0.81,0.36,0.00}{\textbf{#1}}}
\newcommand{\SpecialStringTok}[1]{\textcolor[rgb]{0.31,0.60,0.02}{#1}}
\newcommand{\StringTok}[1]{\textcolor[rgb]{0.31,0.60,0.02}{#1}}
\newcommand{\VariableTok}[1]{\textcolor[rgb]{0.00,0.00,0.00}{#1}}
\newcommand{\VerbatimStringTok}[1]{\textcolor[rgb]{0.31,0.60,0.02}{#1}}
\newcommand{\WarningTok}[1]{\textcolor[rgb]{0.56,0.35,0.01}{\textbf{\textit{#1}}}}
\usepackage{graphicx}
\makeatletter
\newsavebox\pandoc@box
\newcommand*\pandocbounded[1]{% scales image to fit in text height/width
  \sbox\pandoc@box{#1}%
  \Gscale@div\@tempa{\textheight}{\dimexpr\ht\pandoc@box+\dp\pandoc@box\relax}%
  \Gscale@div\@tempb{\linewidth}{\wd\pandoc@box}%
  \ifdim\@tempb\p@<\@tempa\p@\let\@tempa\@tempb\fi% select the smaller of both
  \ifdim\@tempa\p@<\p@\scalebox{\@tempa}{\usebox\pandoc@box}%
  \else\usebox{\pandoc@box}%
  \fi%
}
% Set default figure placement to htbp
\def\fps@figure{htbp}
\makeatother
\setlength{\emergencystretch}{3em} % prevent overfull lines
\providecommand{\tightlist}{%
  \setlength{\itemsep}{0pt}\setlength{\parskip}{0pt}}
\usepackage{bookmark}
\IfFileExists{xurl.sty}{\usepackage{xurl}}{} % add URL line breaks if available
\urlstyle{same}
\hypersetup{
  pdftitle={Introduction to data},
  pdfauthor={Thin Thazin},
  hidelinks,
  pdfcreator={LaTeX via pandoc}}

\title{Introduction to data}
\author{Thin Thazin}
\date{}

\begin{document}
\maketitle

Some define statistics as the field that focuses on turning information
into knowledge. The first step in that process is to summarize and
describe the raw information -- the data. In this lab we explore
flights, specifically a random sample of domestic flights that departed
from the three major New York City airports in 2013. We will generate
simple graphical and numerical summaries of data on these flights and
explore delay times. Since this is a large data set, along the way
you'll also learn the indispensable skills of data processing and
subsetting.

\subsection{Getting started}\label{getting-started}

\subsubsection{Load packages}\label{load-packages}

In this lab, we will explore and visualize the data using the
\textbf{tidyverse} suite of packages. The data can be found in the
companion package for OpenIntro labs, \textbf{openintro}.

Let's load the packages.

\begin{Shaded}
\begin{Highlighting}[]
\FunctionTok{library}\NormalTok{(tidyverse)}
\FunctionTok{library}\NormalTok{(openintro)}
\end{Highlighting}
\end{Shaded}

\subsubsection{The data}\label{the-data}

The \href{http://www.rita.dot.gov/bts/about/}{Bureau of Transportation
Statistics} (BTS) is a statistical agency that is a part of the Research
and Innovative Technology Administration (RITA). As its name implies,
BTS collects and makes transportation data available, such as the
flights data we will be working with in this lab.

First, we'll view the \texttt{nycflights} data frame. Type the following
in your console to load the data:

\begin{Shaded}
\begin{Highlighting}[]
\FunctionTok{data}\NormalTok{(nycflights)}
\end{Highlighting}
\end{Shaded}

The data set \texttt{nycflights} that shows up in your workspace is a
\emph{data matrix}, with each row representing an \emph{observation} and
each column representing a \emph{variable}. R calls this data format a
\textbf{data frame}, which is a term that will be used throughout the
labs. For this data set, each \emph{observation} is a single flight.

To view the names of the variables, type the command

\begin{Shaded}
\begin{Highlighting}[]
\FunctionTok{names}\NormalTok{(nycflights)}
\end{Highlighting}
\end{Shaded}

\begin{verbatim}
##  [1] "year"      "month"     "day"       "dep_time"  "dep_delay" "arr_time" 
##  [7] "arr_delay" "carrier"   "tailnum"   "flight"    "origin"    "dest"     
## [13] "air_time"  "distance"  "hour"      "minute"
\end{verbatim}

This returns the names of the variables in this data frame. The
\textbf{codebook} (description of the variables) can be accessed by
pulling up the help file:

\begin{Shaded}
\begin{Highlighting}[]
\NormalTok{?nycflights}
\end{Highlighting}
\end{Shaded}

One of the variables refers to the carrier (i.e.~airline) of the flight,
which is coded according to the following system.

\begin{itemize}
\tightlist
\item
  \texttt{carrier}: Two letter carrier abbreviation.

  \begin{itemize}
  \tightlist
  \item
    \texttt{9E}: Endeavor Air Inc.
  \item
    \texttt{AA}: American Airlines Inc.
  \item
    \texttt{AS}: Alaska Airlines Inc.
  \item
    \texttt{B6}: JetBlue Airways
  \item
    \texttt{DL}: Delta Air Lines Inc.
  \item
    \texttt{EV}: ExpressJet Airlines Inc.
  \item
    \texttt{F9}: Frontier Airlines Inc.
  \item
    \texttt{FL}: AirTran Airways Corporation
  \item
    \texttt{HA}: Hawaiian Airlines Inc.
  \item
    \texttt{MQ}: Envoy Air
  \item
    \texttt{OO}: SkyWest Airlines Inc.
  \item
    \texttt{UA}: United Air Lines Inc.
  \item
    \texttt{US}: US Airways Inc.
  \item
    \texttt{VX}: Virgin America
  \item
    \texttt{WN}: Southwest Airlines Co.
  \item
    \texttt{YV}: Mesa Airlines Inc.
  \end{itemize}
\end{itemize}

Remember that you can use \texttt{glimpse} to take a quick peek at your
data to understand its contents better.

\begin{Shaded}
\begin{Highlighting}[]
\FunctionTok{glimpse}\NormalTok{(nycflights)}
\end{Highlighting}
\end{Shaded}

\begin{verbatim}
## Rows: 32,735
## Columns: 16
## $ year      <int> 2013, 2013, 2013, 2013, 2013, 2013, 2013, 2013, 2013, 2013, ~
## $ month     <int> 6, 5, 12, 5, 7, 1, 12, 8, 9, 4, 6, 11, 4, 3, 10, 1, 2, 8, 10~
## $ day       <int> 30, 7, 8, 14, 21, 1, 9, 13, 26, 30, 17, 22, 26, 25, 21, 23, ~
## $ dep_time  <int> 940, 1657, 859, 1841, 1102, 1817, 1259, 1920, 725, 1323, 940~
## $ dep_delay <dbl> 15, -3, -1, -4, -3, -3, 14, 85, -10, 62, 5, 5, -2, 115, -4, ~
## $ arr_time  <int> 1216, 2104, 1238, 2122, 1230, 2008, 1617, 2032, 1027, 1549, ~
## $ arr_delay <dbl> -4, 10, 11, -34, -8, 3, 22, 71, -8, 60, -4, -2, 22, 91, -6, ~
## $ carrier   <chr> "VX", "DL", "DL", "DL", "9E", "AA", "WN", "B6", "AA", "EV", ~
## $ tailnum   <chr> "N626VA", "N3760C", "N712TW", "N914DL", "N823AY", "N3AXAA", ~
## $ flight    <int> 407, 329, 422, 2391, 3652, 353, 1428, 1407, 2279, 4162, 20, ~
## $ origin    <chr> "JFK", "JFK", "JFK", "JFK", "LGA", "LGA", "EWR", "JFK", "LGA~
## $ dest      <chr> "LAX", "SJU", "LAX", "TPA", "ORF", "ORD", "HOU", "IAD", "MIA~
## $ air_time  <dbl> 313, 216, 376, 135, 50, 138, 240, 48, 148, 110, 50, 161, 87,~
## $ distance  <dbl> 2475, 1598, 2475, 1005, 296, 733, 1411, 228, 1096, 820, 264,~
## $ hour      <dbl> 9, 16, 8, 18, 11, 18, 12, 19, 7, 13, 9, 13, 8, 20, 12, 20, 6~
## $ minute    <dbl> 40, 57, 59, 41, 2, 17, 59, 20, 25, 23, 40, 20, 9, 54, 17, 24~
\end{verbatim}

The \texttt{nycflights} data frame is a massive trove of information.
Let's think about some questions we might want to answer with these
data:

\begin{itemize}
\tightlist
\item
  How delayed were flights that were headed to Los Angeles?
\item
  How do departure delays vary by month?
\item
  Which of the three major NYC airports has the best on time percentage
  for departing flights?
\end{itemize}

\subsection{Analysis}\label{analysis}

\subsubsection{Departure delays}\label{departure-delays}

Let's start by examing the distribution of departure delays of all
flights with a histogram.

\begin{Shaded}
\begin{Highlighting}[]
\FunctionTok{ggplot}\NormalTok{(}\AttributeTok{data =}\NormalTok{ nycflights, }\FunctionTok{aes}\NormalTok{(}\AttributeTok{x =}\NormalTok{ dep\_delay)) }\SpecialCharTok{+}
  \FunctionTok{geom\_histogram}\NormalTok{()}
\end{Highlighting}
\end{Shaded}

\pandocbounded{\includegraphics[keepaspectratio]{Lab2_intro_to_data_files/figure-latex/hist-dep-delay-1.pdf}}

This function says to plot the \texttt{dep\_delay} variable from the
\texttt{nycflights} data frame on the x-axis. It also defines a
\texttt{geom} (short for geometric object), which describes the type of
plot you will produce.

Histograms are generally a very good way to see the shape of a single
distribution of numerical data, but that shape can change depending on
how the data is split between the different bins. You can easily define
the binwidth you want to use:

\begin{Shaded}
\begin{Highlighting}[]
\FunctionTok{ggplot}\NormalTok{(}\AttributeTok{data =}\NormalTok{ nycflights, }\FunctionTok{aes}\NormalTok{(}\AttributeTok{x =}\NormalTok{ dep\_delay)) }\SpecialCharTok{+}
  \FunctionTok{geom\_histogram}\NormalTok{(}\AttributeTok{binwidth =} \DecValTok{15}\NormalTok{)}
\end{Highlighting}
\end{Shaded}

\pandocbounded{\includegraphics[keepaspectratio]{Lab2_intro_to_data_files/figure-latex/hist-dep-delay-bins-1.pdf}}

\begin{Shaded}
\begin{Highlighting}[]
\FunctionTok{ggplot}\NormalTok{(}\AttributeTok{data =}\NormalTok{ nycflights, }\FunctionTok{aes}\NormalTok{(}\AttributeTok{x =}\NormalTok{ dep\_delay)) }\SpecialCharTok{+}
  \FunctionTok{geom\_histogram}\NormalTok{(}\AttributeTok{binwidth =} \DecValTok{150}\NormalTok{)}
\end{Highlighting}
\end{Shaded}

\pandocbounded{\includegraphics[keepaspectratio]{Lab2_intro_to_data_files/figure-latex/hist-dep-delay-bins-2.pdf}}

\begin{enumerate}
\def\labelenumi{\arabic{enumi}.}
\tightlist
\item
  Look carefully at these three histograms. How do they compare? Are
  features revealed in one that are obscured in another?
\end{enumerate}

\textbf{Insert your answer here} Based on the analysis of all three
histograms, the histogram with no bin width and the one with a bin width
of 15 provide a better representation of the data distribution. In the
histogram with a bin width of 15, both delayed and early departures are
clearly visible, unlike in the histograms with no specified bin width
and with a bin width of 150. Although the histogram with a bin width of
150 shows flight delay counts exceeding 30,000, the distribution is
difficult to interpret due to the wide bins.

If you want to visualize only on delays of flights headed to Los
Angeles, you need to first \texttt{filter} the data for flights with
that destination (\texttt{dest\ ==\ "LAX"}) and then make a histogram of
the departure delays of only those flights.

\begin{Shaded}
\begin{Highlighting}[]
\NormalTok{lax\_flights }\OtherTok{\textless{}{-}}\NormalTok{ nycflights }\SpecialCharTok{\%\textgreater{}\%}
  \FunctionTok{filter}\NormalTok{(dest }\SpecialCharTok{==} \StringTok{"LAX"}\NormalTok{)}
\FunctionTok{ggplot}\NormalTok{(}\AttributeTok{data =}\NormalTok{ lax\_flights, }\FunctionTok{aes}\NormalTok{(}\AttributeTok{x =}\NormalTok{ dep\_delay)) }\SpecialCharTok{+}
  \FunctionTok{geom\_histogram}\NormalTok{()}
\end{Highlighting}
\end{Shaded}

\pandocbounded{\includegraphics[keepaspectratio]{Lab2_intro_to_data_files/figure-latex/lax-flights-hist-1.pdf}}

Let's decipher these two commands (OK, so it might look like four lines,
but the first two physical lines of code are actually part of the same
command. It's common to add a break to a new line after
\texttt{\%\textgreater{}\%} to help readability).

\begin{itemize}
\tightlist
\item
  Command 1: Take the \texttt{nycflights} data frame, \texttt{filter}
  for flights headed to LAX, and save the result as a new data frame
  called \texttt{lax\_flights}.

  \begin{itemize}
  \tightlist
  \item
    \texttt{==} means ``if it's equal to''.
  \item
    \texttt{LAX} is in quotation marks since it is a character string.
  \end{itemize}
\item
  Command 2: Basically the same \texttt{ggplot} call from earlier for
  making a histogram, except that it uses the smaller data frame for
  flights headed to LAX instead of all flights.
\end{itemize}

\phantomsection\label{boxedtext}
\textbf{Logical operators: } Filtering for certain observations
(e.g.~flights from a particular airport) is often of interest in data
frames where we might want to examine observations with certain
characteristics separately from the rest of the data. To do so, you can
use the \texttt{filter} function and a series of \textbf{logical
operators}. The most commonly used logical operators for data analysis
are as follows:

\begin{itemize}
\tightlist
\item
  \texttt{==} means ``equal to''
\item
  \texttt{!=} means ``not equal to''
\item
  \texttt{\textgreater{}} or \texttt{\textless{}} means ``greater than''
  or ``less than''
\item
  \texttt{\textgreater{}=} or \texttt{\textless{}=} means ``greater than
  or equal to'' or ``less than or equal to''
\end{itemize}

You can also obtain numerical summaries for these flights:

\begin{Shaded}
\begin{Highlighting}[]
\NormalTok{lax\_flights }\SpecialCharTok{\%\textgreater{}\%}
  \FunctionTok{summarise}\NormalTok{(}\AttributeTok{mean\_dd   =} \FunctionTok{mean}\NormalTok{(dep\_delay), }
            \AttributeTok{median\_dd =} \FunctionTok{median}\NormalTok{(dep\_delay), }
            \AttributeTok{n         =} \FunctionTok{n}\NormalTok{())}
\end{Highlighting}
\end{Shaded}

\begin{verbatim}
## # A tibble: 1 x 3
##   mean_dd median_dd     n
##     <dbl>     <dbl> <int>
## 1    9.78        -1  1583
\end{verbatim}

Note that in the \texttt{summarise} function you created a list of three
different numerical summaries that you were interested in. The names of
these elements are user defined, like \texttt{mean\_dd},
\texttt{median\_dd}, \texttt{n}, and you can customize these names as
you like (just don't use spaces in your names). Calculating these
summary statistics also requires that you know the function calls. Note
that \texttt{n()} reports the sample size.

\phantomsection\label{boxedtext}
\textbf{Summary statistics: } Some useful function calls for summary
statistics for a single numerical variable are as follows:

\begin{itemize}
\tightlist
\item
  \texttt{mean}
\item
  \texttt{median}
\item
  \texttt{sd}
\item
  \texttt{var}
\item
  \texttt{IQR}
\item
  \texttt{min}
\item
  \texttt{max}
\end{itemize}

Note that each of these functions takes a single vector as an argument
and returns a single value.

You can also filter based on multiple criteria. Suppose you are
interested in flights headed to San Francisco (SFO) in February:

\begin{Shaded}
\begin{Highlighting}[]
\NormalTok{sfo\_feb\_flights }\OtherTok{\textless{}{-}}\NormalTok{ nycflights }\SpecialCharTok{\%\textgreater{}\%}
  \FunctionTok{filter}\NormalTok{(dest }\SpecialCharTok{==} \StringTok{"SFO"}\NormalTok{, month }\SpecialCharTok{==} \DecValTok{2}\NormalTok{)}
\end{Highlighting}
\end{Shaded}

Note that you can separate the conditions using commas if you want
flights that are both headed to SFO \textbf{and} in February. If you are
interested in either flights headed to SFO \textbf{or} in February, you
can use the \texttt{\textbar{}} instead of the comma.

\begin{enumerate}
\def\labelenumi{\arabic{enumi}.}
\setcounter{enumi}{1}
\tightlist
\item
  Create a new data frame that includes flights headed to SFO in
  February, and save this data frame as \texttt{sfo\_feb\_flights}. How
  many flights meet these criteria?
\end{enumerate}

\textbf{Insert your answer here}

\begin{Shaded}
\begin{Highlighting}[]
\NormalTok{sfo\_feb\_flights }\OtherTok{\textless{}{-}}\NormalTok{ nycflights }\SpecialCharTok{\%\textgreater{}\%}
  \FunctionTok{filter}\NormalTok{(dest }\SpecialCharTok{==} \StringTok{"SFO"}\NormalTok{, month }\SpecialCharTok{==} \DecValTok{2}\NormalTok{)}
\end{Highlighting}
\end{Shaded}

68 flights meet these criteria. Does the duplicate flight count???

\begin{enumerate}
\def\labelenumi{\arabic{enumi}.}
\setcounter{enumi}{2}
\tightlist
\item
  Describe the distribution of the \textbf{arrival} delays of these
  flights using a histogram and appropriate summary statistics.
  \textbf{Hint:} The summary statistics you use should depend on the
  shape of the distribution.
\end{enumerate}

\textbf{Insert your answer here}

\begin{Shaded}
\begin{Highlighting}[]
\FunctionTok{ggplot}\NormalTok{(}\AttributeTok{data =}\NormalTok{ sfo\_feb\_flights, }\FunctionTok{aes}\NormalTok{(}\AttributeTok{x =}\NormalTok{ arr\_delay)) }\SpecialCharTok{+} \FunctionTok{geom\_histogram}\NormalTok{(}\AttributeTok{binwidth =} \DecValTok{15}\NormalTok{)}
\end{Highlighting}
\end{Shaded}

\pandocbounded{\includegraphics[keepaspectratio]{Lab2_intro_to_data_files/figure-latex/unnamed-chunk-2-1.pdf}}

\begin{Shaded}
\begin{Highlighting}[]
\NormalTok{sfo\_feb\_flights }\SpecialCharTok{\%\textgreater{}\%}
  \FunctionTok{summarise}\NormalTok{(}\AttributeTok{mean\_sfo =} \FunctionTok{mean}\NormalTok{(arr\_delay),}
            \AttributeTok{median\_sfo =} \FunctionTok{median}\NormalTok{(arr\_delay),}
            \AttributeTok{n =} \FunctionTok{n}\NormalTok{())}
\end{Highlighting}
\end{Shaded}

\begin{verbatim}
## # A tibble: 1 x 3
##   mean_sfo median_sfo     n
##      <dbl>      <dbl> <int>
## 1     -4.5        -11    68
\end{verbatim}

Another useful technique is quickly calculating summary statistics for
various groups in your data frame. For example, we can modify the above
command using the \texttt{group\_by} function to get the same summary
stats for each origin airport:

\begin{Shaded}
\begin{Highlighting}[]
\NormalTok{sfo\_feb\_flights }\SpecialCharTok{\%\textgreater{}\%}
  \FunctionTok{group\_by}\NormalTok{(origin) }\SpecialCharTok{\%\textgreater{}\%}
  \FunctionTok{summarise}\NormalTok{(}\AttributeTok{median\_dd =} \FunctionTok{median}\NormalTok{(dep\_delay), }\AttributeTok{iqr\_dd =} \FunctionTok{IQR}\NormalTok{(dep\_delay), }\AttributeTok{n\_flights =} \FunctionTok{n}\NormalTok{())}
\end{Highlighting}
\end{Shaded}

\begin{verbatim}
## # A tibble: 2 x 4
##   origin median_dd iqr_dd n_flights
##   <chr>      <dbl>  <dbl>     <int>
## 1 EWR          0.5   5.75         8
## 2 JFK         -2.5  15.2         60
\end{verbatim}

Here, we first grouped the data by \texttt{origin} and then calculated
the summary statistics.

\begin{enumerate}
\def\labelenumi{\arabic{enumi}.}
\setcounter{enumi}{3}
\tightlist
\item
  Calculate the median and interquartile range for \texttt{arr\_delay}s
  of flights in in the \texttt{sfo\_feb\_flights} data frame, grouped by
  carrier. Which carrier has the most variable arrival delays?
\end{enumerate}

\textbf{Insert your answer here}

\begin{Shaded}
\begin{Highlighting}[]
\NormalTok{sfo\_feb\_flights }\SpecialCharTok{\%\textgreater{}\%}
  \FunctionTok{group\_by}\NormalTok{(carrier) }\SpecialCharTok{\%\textgreater{}\%}
  \FunctionTok{summarise}\NormalTok{(}\AttributeTok{median\_dd =} \FunctionTok{median}\NormalTok{(arr\_delay), }
            \AttributeTok{iqr\_dd =} \FunctionTok{IQR}\NormalTok{(arr\_delay), }
            \AttributeTok{n\_flights =} \FunctionTok{n}\NormalTok{())}
\end{Highlighting}
\end{Shaded}

\begin{verbatim}
## # A tibble: 5 x 4
##   carrier median_dd iqr_dd n_flights
##   <chr>       <dbl>  <dbl>     <int>
## 1 AA            5     17.5        10
## 2 B6          -10.5   12.2         6
## 3 DL          -15     22          19
## 4 UA          -10     22          21
## 5 VX          -22.5   21.2        12
\end{verbatim}

American Airline AA carrier has the most variable arrival delays.

\subsubsection{Departure delays by
month}\label{departure-delays-by-month}

Which month would you expect to have the highest average delay departing
from an NYC airport?

Let's think about how you could answer this question:

\begin{itemize}
\tightlist
\item
  First, calculate monthly averages for departure delays. With the new
  language you are learning, you could

  \begin{itemize}
  \tightlist
  \item
    \texttt{group\_by} months, then
  \item
    \texttt{summarise} mean departure delays.
  \end{itemize}
\item
  Then, you could to \texttt{arrange} these average delays in
  \texttt{desc}ending order
\end{itemize}

\begin{Shaded}
\begin{Highlighting}[]
\NormalTok{nycflights }\SpecialCharTok{\%\textgreater{}\%}
  \FunctionTok{group\_by}\NormalTok{(month) }\SpecialCharTok{\%\textgreater{}\%}
  \FunctionTok{summarise}\NormalTok{(}\AttributeTok{mean\_dd =} \FunctionTok{mean}\NormalTok{(dep\_delay)) }\SpecialCharTok{\%\textgreater{}\%}
  \FunctionTok{arrange}\NormalTok{(}\FunctionTok{desc}\NormalTok{(mean\_dd))}
\end{Highlighting}
\end{Shaded}

\begin{verbatim}
## # A tibble: 12 x 2
##    month mean_dd
##    <int>   <dbl>
##  1     7   20.8 
##  2     6   20.4 
##  3    12   17.4 
##  4     4   14.6 
##  5     3   13.5 
##  6     5   13.3 
##  7     8   12.6 
##  8     2   10.7 
##  9     1   10.2 
## 10     9    6.87
## 11    11    6.10
## 12    10    5.88
\end{verbatim}

\begin{enumerate}
\def\labelenumi{\arabic{enumi}.}
\setcounter{enumi}{4}
\tightlist
\item
  Suppose you really dislike departure delays and you want to schedule
  your travel in a month that minimizes your potential departure delay
  leaving NYC. One option is to choose the month with the lowest mean
  departure delay. Another option is to choose the month with the lowest
  median departure delay. What are the pros and cons of these two
  choices?
\end{enumerate}

\textbf{Insert your answer here}

\subsubsection{On time departure rate for NYC
airports}\label{on-time-departure-rate-for-nyc-airports}

Suppose you will be flying out of NYC and want to know which of the
three major NYC airports has the best on time departure rate of
departing flights. Also supposed that for you, a flight that is delayed
for less than 5 minutes is basically ``on time.''\,'' You consider any
flight delayed for 5 minutes of more to be ``delayed''.

In order to determine which airport has the best on time departure rate,
you can

\begin{itemize}
\tightlist
\item
  first classify each flight as ``on time'' or ``delayed'',
\item
  then group flights by origin airport,
\item
  then calculate on time departure rates for each origin airport,
\item
  and finally arrange the airports in descending order for on time
  departure percentage.
\end{itemize}

Let's start with classifying each flight as ``on time'' or ``delayed''
by creating a new variable with the \texttt{mutate} function.

\begin{Shaded}
\begin{Highlighting}[]
\NormalTok{nycflights }\OtherTok{\textless{}{-}}\NormalTok{ nycflights }\SpecialCharTok{\%\textgreater{}\%}
  \FunctionTok{mutate}\NormalTok{(}\AttributeTok{dep\_type =} \FunctionTok{ifelse}\NormalTok{(dep\_delay }\SpecialCharTok{\textless{}} \DecValTok{5}\NormalTok{, }\StringTok{"on time"}\NormalTok{, }\StringTok{"delayed"}\NormalTok{))}
\end{Highlighting}
\end{Shaded}

The first argument in the \texttt{mutate} function is the name of the
new variable we want to create, in this case \texttt{dep\_type}. Then if
\texttt{dep\_delay\ \textless{}\ 5}, we classify the flight as
\texttt{"on\ time"} and \texttt{"delayed"} if not, i.e.~if the flight is
delayed for 5 or more minutes.

Note that we are also overwriting the \texttt{nycflights} data frame
with the new version of this data frame that includes the new
\texttt{dep\_type} variable.

We can handle all of the remaining steps in one code chunk:

\begin{Shaded}
\begin{Highlighting}[]
\NormalTok{nycflights }\SpecialCharTok{\%\textgreater{}\%}
  \FunctionTok{group\_by}\NormalTok{(origin) }\SpecialCharTok{\%\textgreater{}\%}
  \FunctionTok{summarise}\NormalTok{(}\AttributeTok{ot\_dep\_rate =} \FunctionTok{sum}\NormalTok{(dep\_type }\SpecialCharTok{==} \StringTok{"on time"}\NormalTok{) }\SpecialCharTok{/} \FunctionTok{n}\NormalTok{()) }\SpecialCharTok{\%\textgreater{}\%}
  \FunctionTok{arrange}\NormalTok{(}\FunctionTok{desc}\NormalTok{(ot\_dep\_rate))}
\end{Highlighting}
\end{Shaded}

\begin{verbatim}
## # A tibble: 3 x 2
##   origin ot_dep_rate
##   <chr>        <dbl>
## 1 LGA          0.728
## 2 JFK          0.694
## 3 EWR          0.637
\end{verbatim}

\begin{enumerate}
\def\labelenumi{\arabic{enumi}.}
\setcounter{enumi}{5}
\tightlist
\item
  If you were selecting an airport simply based on on time departure
  percentage, which NYC airport would you choose to fly out of?
\end{enumerate}

You can also visualize the distribution of on on time departure rate
across the three airports using a segmented bar plot.

\begin{Shaded}
\begin{Highlighting}[]
\FunctionTok{ggplot}\NormalTok{(}\AttributeTok{data =}\NormalTok{ nycflights, }\FunctionTok{aes}\NormalTok{(}\AttributeTok{x =}\NormalTok{ origin, }\AttributeTok{fill =}\NormalTok{ dep\_type)) }\SpecialCharTok{+}
  \FunctionTok{geom\_bar}\NormalTok{()}
\end{Highlighting}
\end{Shaded}

\pandocbounded{\includegraphics[keepaspectratio]{Lab2_intro_to_data_files/figure-latex/viz-origin-dep-type-1.pdf}}

\textbf{Insert your answer here}

\begin{center}\rule{0.5\linewidth}{0.5pt}\end{center}

\subsection{More Practice}\label{more-practice}

\begin{enumerate}
\def\labelenumi{\arabic{enumi}.}
\setcounter{enumi}{6}
\tightlist
\item
  Mutate the data frame so that it includes a new variable that contains
  the average speed, \texttt{avg\_speed} traveled by the plane for each
  flight (in mph). \textbf{Hint:} Average speed can be calculated as
  distance divided by number of hours of travel, and note that
  \texttt{air\_time} is given in minutes.
\end{enumerate}

\textbf{Insert your answer here}

\begin{enumerate}
\def\labelenumi{\arabic{enumi}.}
\setcounter{enumi}{7}
\tightlist
\item
  Make a scatterplot of \texttt{avg\_speed} vs.~\texttt{distance}.
  Describe the relationship between average speed and distance.
  \textbf{Hint:} Use \texttt{geom\_point()}.
\end{enumerate}

\textbf{Insert your answer here}

\begin{enumerate}
\def\labelenumi{\arabic{enumi}.}
\setcounter{enumi}{8}
\tightlist
\item
  Replicate the following plot. \textbf{Hint:} The data frame plotted
  only contains flights from American Airlines, Delta Airlines, and
  United Airlines, and the points are \texttt{color}ed by
  \texttt{carrier}. Once you replicate the plot, determine (roughly)
  what the cutoff point is for departure delays where you can still
  expect to get to your destination on time.
\end{enumerate}

\pandocbounded{\includegraphics[keepaspectratio]{Question9Plot.png}}

\textbf{Insert your answer here}

\end{document}
